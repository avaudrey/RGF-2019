\documentclass[a4paper,11pt]{scrartcl}

\usepackage[french]{babel}
\usepackage{fontspec,xunicode,xltxtra}
\usepackage[width=17cm,height=25cm]{geometry}
\usepackage{amsmath,amssymb,amsthm}
\usepackage[auth-sc]{authblk}
\usepackage{graphicx,xcolor}
\usepackage{csquotes}
\usepackage[style=numeric-comp, backend=biber,sorting=none,url=false,maxbibnames=99]{biblatex}
\bibliography{biblio}
\usepackage{hyperref}
\hypersetup{colorlinks,pdftitle={},pdfauthor={A.~Vaudrey},citecolor=blue,filecolor=blue,linkcolor=blue,urlcolor=blue}

\renewcommand*{\Affilfont}{\small\normalfont}

\title{Application pratique de l'analyse exergétique aux systèmes de traitement de l'air : le problème du choix de l'état de référence de l'eau}
\author[1,2]{J. Chacon Chauca}
\author[3]{A. M. Quintanilla Muñoz}
\author[2]{A. Vaudrey\thanks{Auteur correspondant : ECAM Lyon, Pôle Énergétique/LabECAM, 40 montée Saint-Barthélemy, 69321 Lyon CEDEX 05. Tél : +33 (0)4 72 77 06 30, \href{mailto:alexandre.vaudrey@ecam.fr}{alexandre.vaudrey@ecam.fr}, \href{http://orcid.org/0000-0002-8613-774X}{ORCID iD: 0000-0002-8613-774X}}\textsuperscript{,}}
\affil[1]{Université de Lyon, ECAM Lyon, INSA-Lyon, LabECAM, F-69005, France.}
\affil[2]{Pontificia Universidad Cat\'olica del Per\'u (PUCP), Laboratorio
de Energ\'ia, Lima, Per\'u.}
\affil[3]{Département Énergie, Institut FEMTO-ST, UMR 6174, CNRS, Université de Bourgogne Franche-Comté, Belfort, France.}

\begin{document}
\maketitle

\begin{abstract}
	Lorsqu'elle est appliquée aux systèmes de traitement de l'air, l'analyse
	exergétique se heurte à une complication liée au choix de l'état de
	référence de l'eau. Ce choix, qui peut de prime abord ressembler à un
	simple problème de convention d'écriture, peut cependant déboucher en
	pratique sur des résultats très différents, qui peuvent donner un image
	fausse des performances réelles des systèmes analysés.
\end{abstract}

\section{Introduction}

Sa simplicité et sa pertinence ont fait passer au fil des décennies
l'\emph{analyse exergétique} du statut de concept théorique à celui d'outil
pratique d'ingénierie, voire même de support de vulgarisation \cite{SV-2013}.
Dans certains cas cependant, son utilisation n’est pas si naturelle et requiert
un travail de clarification supplémentaire. La manipulation de températures
inférieures à celle de l’ambiance \cite{RGT-035-0651} ou l’exploitation de
l’énergie solaire \cite{EPL-104-40008} sont des exemples bien connus de telles
difficultés. Pour ce qui concerne les applications de traitement de l’air, le
choix de \emph{l’état de référence de l’eau} (sur lequel nous allons revenir en
détails dans la suite) est justement un de ceux ayant les plus lourdes
conséquences sur les résultats finalement obtenus. De ce choix peuvent ainsi
découler des opinions très différentes quant aux performances réelles des
systèmes analysés. Nous allons aborder dans la suite de cet article l’importance
d’un tel choix et, sur la base de résultats expérimentaux, présenter un aperçu
de ses conséquences sur les performances estimées d’un système de traitement de
l’air. L’objectif visé ici est d’aboutir à une convention relative à cet état de
référence de l’eau, qui permette aux praticiens d’appliquer sereinement
l’analyse exergétique aux systèmes de traitement de l’air, et de bénéficier
pleinement de l’efficacité bien connue d’une telle approche.

\section{L'analyse exergétique}

\subsection{L'exergie} Une des conséquences, ou \emph{énoncé}, du second
principe de la thermodynamique, est qu'une quantité de chaleur quelconque ne
peut être échangée entre deux milieux différents si ceux-ci sont initialement à
deux températures différentes. Cette quantité de chaleur peut ensuite être
partiellement transformée en travail par le truchement de la variation de
pression, puis de volume, de l'un de ces deux milieux. Il est possible aussi,
via un effet osmotique par exemple, d'utiliser la différence de compositions
chimiques entre deux milieux pour produire du travail. Ainsi, la quantité de
travail qu'il est possible d'extraire d'une quantité donnée de matière est liée
à sa différence de \emph{température}, de \emph{pression} et de
\emph{composition chimique} dont elle dispose avec son milieu ambiant, milieu
avec lequel elle finira toujours par atteindre un \emph{équilibre
thermodynamique}.  (Cet équilibre thermodynamique est justement traduit en
pratique par une égalité de la température, de la pression et de la composition
chimique entre la quantité de matière en question et son milieu extérieur.)
C'est à partir de ce constat que l'on peut définir le concept d'\emph{exergie}.

\paragraph{Définition} L'\emph{exergie} est la quantité maximale de travail
qu'il est possible d'extraire d'une quantité de matière donnée lorsque cette
dernière évolue d'un état initial quelconque à un état final qui est celui de
son milieu extérieur.

\bigskip

L'\emph{état} en question est représenté concrètement par un jeu de variables
d'état: la \emph{température}, la \emph{pression} et les \emph{concentrations}
respectives des éléments qui composent le système concerné. D'après les
explications précédentes, une telle exergie est la somme de trois composantes :
\begin{itemize}
	\item l'\emph{exergie thermique} $Ex_{\text{therm}}$, due à la
		différence entre la température $T$ du système de capacité
		thermique $C$, et celle notée $T_0$ du milieu
		extérieur\footnote{Nous adopterons dans la suite la
		convention d'écriture la plus courante dans le domaine de
		l'exergie, qui consiste à noter d'un indice $0$ les grandeurs
		propres au milieu extérieur.}, est donnée
		par~\cite{entropie-157-0013-0020}: \begin{equation}
			Ex_{\text{therm}} = C \cdot (T-T_0) \cdot
			\Theta_{\text{ml},0} \label{eq:exergie-th-0}
		\end{equation} $\Theta=1-T_0/T$ est appelé facteur de
		\textsc{Carnot} --- parce qu'il est égal au rendement (de
		\textsc{Carnot}) d'un moteur thermique idéal fonctionnant entre
		les deux températures $T_0$ et $T$ --- et s'écrit dans le cas de
		l'équation \eqref{eq:exergie-th-0}, sous la forme:
		\begin{equation}
			\Theta_{\text{ml},0} = 1 - \frac{T_0}{T_{\text{ml},0}}
			\text{ avec } T_{\text{ml},0} =
			\frac{T-T_0}{\ln(T/T_0)}
		\end{equation} $T_{\text{ml},0}$ est ici la bien connue
		\emph{température moyenne logarithmique} entre $T$ et $T_0$;
	\item l'\emph{exergie mécanique} $Ex_{\text{méca}}$, issue de la
		différence de pression entre l'intérieur du système et le milieu
		extérieur, et dont l'expression n'est autre que celle du travail
		de détente isotherme à température $T_0$ de $p$ à
		$p_0$~\cite{TI-BE8013} : \begin{equation}
			Ex_{\text{méca}} = n \cdot \textsf{R} \cdot T_0 \cdot
			\ln\left( \frac{p}{p_0} \right) \label{eq:exergie-m-0}
		\end{equation} $n$ est ici la quantité molaire de matière mise
		en jeu, et $\mathsf{R}$ est la constante universelle des gaz
		parfaits;
	\item l'\emph{exergie chimique} $Ex_{\text{chim}}$ qui existe tant que
		la composition chimique interne du système est différente de
		celle de son milieu extérieur. Elle s'exprime sous la forme
		de~\cite{TI-BE8015} : \begin{equation}
			Ex_{\text{chim}} = \sum_i n_i \cdot (\mu_i - \mu_{i,0})
		\end{equation} Les indices $i$ correspondant à chacun des
		composants du système, les $\mu_i$ étant leurs \emph{potentiels
		chimiques} respectifs, pris à la température $T$ et à la
		\emph{pression partielle} $p_i$ à l'intérieur du système, et à
		la température $T_0$ et la pression partielle $p_{i,0}$ dans le
		milieu extérieur.
\end{itemize}

%Dans le cas d'un bâtiment par exemple, la nécessité de chauffer ou de refroidir
%l'air de ventilation, de l'humidifier ou de le déshumidifier, dépend de la
%température et de l'humidité extérieure. Réciproquement à la définition
%précédente, l'exergie est aussi la quantité minimale de travail qu'il faut
%échanger avec une quantité ou un flux d'énergie, pour les faire passer d'un état
%d'équilibre avec le milieu extérieur à un état d’état d'énergie plus élevée. Par
%état, on entend en pratique la température, la pression et la composition
%chimique.

%\textsl{Les applications de traitement de l'air constituent un exemple très intéressant d'utilisation de l'analyse exergétique parce qu'elles consomment justement de l'air venant de l’extérieur, et dont l'exergie spécifique est nulle par définition.}
\printbibliography

\end{document}
