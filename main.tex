\documentclass[a4paper,11pt]{scrartcl}

\usepackage[french]{babel}
\usepackage{fontspec,xunicode,xltxtra}
\usepackage[width=17cm,height=25cm]{geometry}
\usepackage{amsmath,amssymb,amsthm,gensymb,xspace}
\usepackage[auth-sc]{authblk}
\usepackage{graphicx,xcolor}
\usepackage{csquotes}
\usepackage{booktabs,ltablex,numprint,multirow,colortbl}
\usepackage[style=numeric-comp, backend=biber, sorting=none, url=false, maxbibnames=99]{biblatex}
\bibliography{biblio}
\usepackage{multicol}
\usepackage{hyperref}
\hypersetup{colorlinks,pdftitle={},pdfauthor={A.~Vaudrey},citecolor=blue,filecolor=blue,linkcolor=blue,urlcolor=blue}

\newcommand{\dC}[1]{\numprint[\degree{\mathrm{C}}]{#1}}
\renewcommand*{\Affilfont}{\small\normalfont}
\renewcommand*{\Authfont}{\normalsize\scshape}

\title{Application pratique de l'analyse exergétique aux systèmes de traitement
de l'air : les pièges à éviter}
\author[1,2]{J. Chacon Chauca}
\author[3]{A. M. Quintanilla Muñoz}
\author[1,2]{A. Vaudrey\thanks{Auteur correspondant : ECAM Lyon, Pôle Énergétique/LabECAM, 40 montée Saint-Barthélemy, 69321 Lyon CEDEX 05. Tél : +33 (0)4 72 77 06 30, \href{mailto:alexandre.vaudrey@ecam.fr}{alexandre.vaudrey@ecam.fr}, \href{http://orcid.org/0000-0002-8613-774X}{ORCID iD: 0000-0002-8613-774X}}\textsuperscript{,}}
\affil[1]{Université de Lyon, ECAM Lyon, INSA-Lyon, LabECAM, F-69005, France.}
\affil[2]{Pontificia Universidad Cat\'olica del Per\'u (PUCP), Laboratorio
de Energ\'ia, Lima, Pérou.}
\affil[3]{Département Énergie, Institut FEMTO-ST, UMR 6174, CNRS, Université de Bourgogne Franche-Comté, Belfort, France.}

\begin{document}
\maketitle

\begin{abstract}
	L'\emph{analyse exergétique} est un outil d'ingénierie à la fois simple
	et efficace, qui fournit une image précise des performances énergétiques
	des systèmes auxquels elle est appliquée. Son utilisation concrète sur
	des installations en fonctionnement, \textsl{a fortiori} s'il s'agit de
	systèmes de traitement de l'air, nécessite cependant de savoir
	contourner certaines difficultés à la fois conceptuelles et
	expérimentales. Cet article vise justement à proposer aux praticiens
	intéressés des chemins permettant de contourner ces difficultés, et de
	profiter pleinement de tous les avantages que l'analyse exergétique peut
	apporter.
\end{abstract}

\tableofcontents

\section{Introduction}

Sa simplicité et sa pertinence ont fait passer au fil des décennies
l'\emph{analyse exergétique} du statut de concept théorique à celui d'outil
pratique d'ingénierie, et même de sujet de vulgarisation \cite{SV-2013}. Comme
beaucoup d'autres types d'installations, les systèmes de traitement de l'air ont
été l'objet depuis la fin des années 1970 d'études basées sur l'\emph{exergie},
qui ont clairement démontré la pertinence d'une telle
approche~\cite{Gaggioli-1978,ASHRAE-1979,ECM-021-0065}. Cependant, même si nombre d'articles et même d'ouvrages~\cite{Dincer-2015}, ont été publiés depuis, très peu font clairement mention des difficultés pratiques auxquelles sont confrontés
celles et ceux qui souhaitent appliquer l'analyse exergétique aux installations
de traitement de l'air.

Dans certains cas en effet, l'utilisation de l'approche exergétique n’est pas si
naturelle, et requiert un travail de clarification supplémentaire. La
manipulation de températures inférieures à celle de l’ambiance
\cite{RGT-035-0651} ou l’exploitation de l’énergie solaire \cite{EPL-104-40008}
sont des exemples bien connus de telles difficultés. Pour ce qui concerne les
applications de traitement de l’air, le choix de \emph{l’état de référence de
l’eau} (sur lequel nous allons revenir en détails dans la suite) est un exemple
de difficulté ayant les plus lourdes conséquences sur les résultats finalement
obtenus. De ce choix peuvent ainsi découler des opinions très différentes quant
aux performances réelles des systèmes analysés. 

Nous allons ainsi revenir dans la suite sur le concept d'exergie et sur son application à l'air humide, puis présenter les principales difficultés pratiques dans son application à l'analyse des systèmes de traitement de l'air. Nous proposerons finalement des pistes de réflexion visant à faire de cette approche un outil vraiment courant lors de la conception et de l'exploitation de telles installations.

%Nous allons aborder dans la suite de cet article l'importance d’un tel choix
%et, sur la base de résultats expérimentaux, présenter un aperçu de ses
%conséquences sur les performances estimées d’un système de traitement de l’air.
%L’objectif visé ici est d’aboutir à une convention relative à cet état de
%référence de l’eau, qui permette aux praticiens d’appliquer sereinement
%l’analyse exergétique aux systèmes de traitement de l’air, et de bénéficier
%pleinement de l’efficacité bien connue d’une telle approche.

\section{L'analyse exergétique}

\subsection{L'exergie} Une des conséquences, ou \emph{énoncé}, du second
principe de la thermodynamique, est qu'une quantité de chaleur quelconque ne
peut être échangée entre deux milieux différents que si ceux-ci sont
initialement à deux températures différentes. Cette quantité de chaleur peut
ensuite être partiellement transformée en travail grâce, par exemple, à la
variation de pression, puis de volume, de l'un de ces deux milieux. Il est
possible aussi, via un effet osmotique par exemple, d'utiliser la différence de
compositions chimiques entre deux milieux pour produire du travail. Ainsi, la
quantité de travail qu'il est possible d'extraire d'une quantité donnée de
matière est liée à sa différence de \emph{température}, de \emph{pression} et de
\emph{composition chimique} avec son milieu ambiant, milieu avec lequel elle
finira toujours par atteindre un \emph{équilibre thermodynamique}. (Cet
équilibre thermodynamique est justement traduit en pratique par une égalité de
la température, de la pression et de la composition chimique entre la quantité
de matière en question et son milieu extérieur.) C'est à partir de ce constat
que l'on peut définir le concept d'\emph{exergie}.

\paragraph{Définition} \textsl{L'exergie est la quantité maximale de travail
qu'il est possible d'extraire d'une quantité de matière donnée lorsque cette
dernière évolue d'un état initial quelconque à un état final qui est celui de
son milieu extérieur.}

\bigskip

L'\emph{état} dont il est question dans cette définition est représenté
concrètement par un jeu de variables d'état: la \emph{température}, la
\emph{pression} et les \emph{concentrations} de chacun des éléments qui
composent le système concerné~\cite{TI-BE8015}. Le fait que l'exergie soit une quantité \emph{maximale} de travail extrait signifie que les processus en question sont \emph{ideaux}, ou \emph{réversibles}, et qu'ils peuvent donc être décrits par une approche thermodynamique. D'après les explications
précédentes, une telle exergie peut être exprimée comme la somme de trois
composantes : \begin{itemize}
	\item l'\emph{exergie thermique} $Ex_{\text{therm}}$, due à la
		différence entre la température $T$ du système de capacité
		thermique $C$ en $\mathrm{J/K}$, et celle notée $T_0$ du milieu
		extérieur\footnote{Nous adopterons dans la suite la
		convention d'écriture la plus courante dans le domaine de
		l'exergie, qui consiste à noter d'un indice $0$ les grandeurs
		propres au milieu extérieur. Une table des notations est par
		ailleurs présentée en
		page~\pageref{table-notations}.}~\cite{entropie-157-0013-0020} :
		\begin{equation}
			Ex_{\text{therm}} = C \cdot (T-T_0) \cdot
			\Theta_{\text{ml},0} \label{eq:exergie-th-0}
		\end{equation} $\Theta=1-T_0/T$ est un paramètre sans dimension
		appelé facteur de \textsc{Carnot} --- parce que son expression
		est identique à celle du rendement (de \textsc{Carnot}) d'un
		moteur thermique idéal fonctionnant entre les deux températures
		$T_0$ et $T$ --- et s'écrit dans le cas de l'équation
		\eqref{eq:exergie-th-0}, sous la forme: \begin{equation}
			\Theta_{\text{ml},0} = 1 - \frac{T_0}{T_{\text{ml},0}}
			\text{ avec } T_{\text{ml},0} =
			\frac{T-T_0}{\ln(T/T_0)}
		\end{equation} $T_{\text{ml},0}$ est ici la \emph{température
		moyenne logarithmique} de $T$ et $T_0$;
	\item l'\emph{exergie mécanique} $Ex_{\text{méca}}$, issue de la
		différence de pression entre l'intérieur et l'extérieur du
		système, et dont l'expression n'est autre que celle du travail
		de détente isotherme (à température $T_0$) de $p$ à
		$p_0$~\cite{TI-BE8013} : \begin{equation}
			Ex_{\text{méca}} = n \cdot \textsf{R} \cdot T_0 \cdot
			\ln\left( \frac{p}{p_0} \right) \label{eq:exergie-m-0}
		\end{equation} $n$ est ici la quantité molaire de matière mise
		en jeu, et $\mathsf{R}$ est la constante universelle des gaz
		parfaits\footnote{On rappelle au passage que $\mathsf{R} =
		\numprint[J/(mol \cdot K)]{8.3144486}$.};
	\item l'\emph{exergie chimique} $Ex_{\text{chim}}$ qui existe tant que
		la composition chimique interne du système est différente de
		celle de son milieu extérieur. Elle s'exprime sous la forme
		de~\cite{TI-BE8015} : \begin{equation}
			Ex_{\text{chim}} = \sum_i n_i \cdot (\mu_i - \mu_{i,0})
		\end{equation} Les indices $i$ correspondent à chacun des
		composants du système, représentant une quantité molaire $n_i$
		(telle bien entendu que $\sum_i n_i = n$ avec la quantité $n$
		qui est elle de la relation \eqref{eq:exergie-m-0}). Les $\mu_i$
		sont les \emph{potentiels chimiques} de chacun de ces
		composants, pris à la température $T$ et à la \emph{pression
		partielle} $p_i$ à l'intérieur du système, et à la température
		$T_0$ et la pression partielle $p_{i,0}$ dans le milieu
		extérieur.
\end{itemize} Ainsi, tout système thermodynamique composé des pressions
partielles $p_i$ de différents éléments, le tout à la température $T$ et à la
pression $p$ dispose, s'il se trouve dans un milieu extérieur pour lequel ces
paramètres sont respectivement $p_{i,0}$, $T_0$ et $p_0$, dispose qu'une
quantité d'énergie qu'il peut potentiellement échanger sous forme de travail
avec ce dernier, son \emph{exergie}, qui s'écrit : \begin{equation}
	Ex = Ex_{\text{therm}} + Ex_{\text{méca}} + Ex_{\text{chim}} \qquad
	[\mathrm{J}]
\end{equation} À noter que, comme pour l'énergie, il est possible de définir un
taux d'échange d'exergie, ou \emph{puissance exergétique}, notée généralement
$\dot{E}x$ et exprimée en W.

% Courte explication sur l'utilité de l'exergie, de manière générale.

\subsection{L'exergie de l'air humide} L'application du concept d'exergie au cas
spécifique de l'\emph{air humide}, telle que formalisée initialement par
\citeauthor{ASHRAE-1979}~\cite{ASHRAE-1979}, consiste ainsi à appliquer le
raisonnement précédent au mélange d'air sec et de vapeur d'eau, considéré comme
un mélange de gaz parfaits. En rapportant chaque quantité d'énergie à la masse
d'air sec $m_a$ mise en jeu, comme c'est l'usage dans le domaine de l'air
humide~\cite{TI-B2230}, on arrive à l'exergie spécifique thermique de l'air
humide suivante, d'après \eqref{eq:exergie-th-0} : \begin{equation}
	ex_{\text{therm}} = \frac{Ex_{\text{therm}}}{m_a} = \left( c_{p,a} +
	\omega \cdot c_{p,v} \right) \cdot (T-T_0) \cdot \Theta_{\text{ml},0}
	\label{eq:exergie-th-1}
\end{equation} $c_{p,a}$ et $c_{p,v}$ sont les capacités thermiques spécifiques
à pression constante de l'air sec et de la vapeur d'eau (dont les valeurs
numériques standards sont détaillées dans le
tableau~\ref{tab:valeurs-reference}), et $\omega$ est l'\emph{humidité spécifique} définie par : \begin{equation}
	\omega = \frac{m_v}{m_a} \label{eq:def-omega}
\end{equation} En utilisant les mêmes notations, l'exergie mécanique spécifique
de l'air humide s'écrit, d'après \eqref{eq:exergie-m-0} : \begin{equation}
	ex_{\text{méca}} = \frac{Ex_{\text{méca}}}{m_a} = (\mathsf{r}_a + \omega
	\cdot \mathsf{r}_v) \cdot T_0 \cdot \ln\left( \frac{p}{p_0} \right)
	\label{eq:exergie-m-1}
\end{equation} Les termes $\mathsf{r}_i$ sont ici les \emph{constantes spécifiques massiques} de chacun des composants du mélange gazeux, telles que $\mathsf{r}_i =
\mathsf{R}/M_i$ avec $M_i$ la masse molaire de chacun de ces composants.
\begin{table}[b]
	\caption{Valeurs standards des propriétés physiques des deux composants
	de l'air humide, d'après \citeauthor{TI-B2230}~\cite{TI-B2230}.}
	\centering
	\begin{tabular}{ln{2}{5}n{2}{5}}
		\toprule
		& \textbf{Air sec} & \textbf{Vapeur d'eau} \\
		\midrule
		Masse molaire $M$, en $[\mathrm{g/mol}]$ & $28.965$ & $18.016$ \\
		Constante spécifique massique $\mathsf{r}$, en $\mathrm{kJ/(kg \cdot
		K)}$ & $0.28705$ & $0.46151$ \\
		Capacité thermique spécifique $c_p$, en $\mathrm{kJ/(kg \cdot
		K)}$ & $1.006$ & $1.827$  \\
		Rapport des masses molaires $\alpha = M_v/M_a$ &
		\multicolumn{2}{c}{$\numprint{0.622}$} \\
		\bottomrule
	\end{tabular}
	\label{tab:valeurs-reference}
\end{table} Lorsqu'elle est appliquée à l'air humide, l'exergie spécifique
chimique fait tout d'abord apparaître les pressions partielles de l'air sec et
de la vapeur d'eau sous une forme très proche de celle de l'exergie mécanique
\eqref{eq:exergie-m-0}, comme expliqué par exemple dans \cite{TI-BE8015}. Pour des raisons pratiques, il est alors usuel de remplacer ces pressions partielles par le paramètre qui permet en pratique à
lui seul de décrire la proportion d'air sec et de vapeur d'eau dans l'air
humide : l'humidité spécifique $\omega$. On arrive alors à l'expression :
\begin{equation}
	ex_{\text{chim}} = T_0 \cdot \left[ \omega \cdot \mathsf{r}_v \cdot
	\ln\left( \frac{\omega}{\omega_0} \right) - (\mathsf{r}_a + \omega \cdot
	\mathsf{r}_v) \cdot \ln\left( \frac{\omega+\alpha}{\omega_0+\alpha}
	\right) \right] \label{eq:exergie-chim-1}
\end{equation} $\alpha = \mathsf{r}_a/\mathsf{r}_v = M_v/M_a \simeq
\numprint{0.622}$ est ici le rapport de la masse molaire de la vapeur d'eau à
celle de l'air sec. Une démonstration complète de l'obtention de cette formule
est présentée, quoique sous une forme légèrement différente, dans la référence
\cite[page 213]{Bejan-2006}. En effet, en fonction des ouvrages ou articles consultés,
l'exergie spécifique thermique \eqref{eq:exergie-th-1} est parfois exprimée sans
faire appel spécifiquement à un facteur de \textsc{Carnot}, comme dans
\cite{ASHRAE-1979,Bejan-2006}. Parfois aussi, l'humidité spécifique $\omega$ est
remplacée par son homologue molaire, notée souvent $\tilde{\omega} = n_v/n_a$.
Quoi qu'il en soit, l'exergie spécifique massique de l'air humide peut s'écrire,
d'après \eqref{eq:exergie-th-1}, \eqref{eq:exergie-m-1} et
\eqref{eq:exergie-chim-1}, sous la forme de la somme : \begin{equation}
	\begin{split}
		ex = & \left( c_{p,a} + \omega \cdot c_{p,v} \right) \cdot
		(T-T_0) \cdot \Theta_{\text{ml},0} + (\mathsf{r}_a + \omega
		\cdot \mathsf{r}_v) \cdot T_0 \cdot \ln\left( \frac{p}{p_0}
		\right) \\
		& + T_0 \cdot \left[ \omega \cdot \mathsf{r}_v \cdot \ln\left(
		\frac{\omega}{\omega_0} \right) - (\mathsf{r}_a + \omega \cdot
		\mathsf{r}_v) \cdot \ln\left(
		\frac{\omega+\alpha}{\omega_0+\alpha} \right) \right]
	\end{split} \label{eq:exergie-specifique}
\end{equation} La différence de température $T-T_0$ n'a ainsi d'influence que
sur la fraction \og thermique \fg de l'exergie spécifique, tout comme la
différence de pression $p-p_0$ n'a d'effet que sur sa fraction \og mécanique
\fg. Une quelconque variation de l'humidité spécifique $\omega-\omega_0$ au
contraire, influence tout les types d'exergie de l'air humide. Un exemple
d'influence de la température et de la pression sur l'exergie spécifique de
l'air humide est présenté dans le tableau~\ref{tab:exemple-calcul-1}.
\begin{table}[t]
	\caption{Partant d'un air extérieur à $T_0 = \dC{0}$, $p_0 =
	\numprint[atm]{1}$ et $\varphi_0 = 80\%$ en hiver, puis d'un autre air
	extérieur à $T_0 = \dC{30}$, $p_0 = \numprint[atm]{1}$ et $\varphi_0 =
	40\%$ en été, $\Delta T = T-T_0$ et $\Delta p = p-p_0$ sont les
	variations de température et de pression auxquelles l'air humide doit
	être soumis pour faire varier son exergie spécifique, telle que définie
	par \eqref{eq:exergie-specifique}, de la valeur indiquée $\Delta ex$.
	$\mathrm{mm}_{\text{CE}}$ est ici le millimètre de colonne d'eau.}
	\centering
	\begin{tabular}{n{1}{2}n{2}{2}n{3}{2}n{2}{2}n{3}{2}}
		\toprule
		& \multicolumn{2}{c}{Hiver} & \multicolumn{2}{c}{Été} \\
		{$\Delta ex \,[\mathrm{kJ/kg}]$} & {$\Delta
		T\,[\degree\mathrm{C}]$} & {$\Delta
		p\,[\mathrm{mm}_{\text{CE}}]$} & {$\Delta
		T\,[\degree\mathrm{C}]$} & {$\Delta
		p\,[\mathrm{mm}_{\text{CE}}]$}  \\
		\midrule
		$0.1$ & $7.42$ & $13.12$ & $-7.62$ & $11.68$ \\
		$0.25$ & $11.79$ & $32.84$ & $-12.00$ & $29.23$ \\
		$0.5$ & $16.76$ & $65.78$ & $-16.87$ & $58.53$ \\
		$0.75$ & $20.62$ & $98.82$ & $-20.57$ & $87.92$ \\
		$1$ & $23.90$ & $131.97$ & $-23.67$ & $117.40$ \\
		$1.5$ & $29.46$ & $198.59$ & $-28.81$ & $176.60$ \\
		$2$ & $34.20$ & $265.63$ & $-33.10$ & $263.13$ \\
		\bottomrule
	\end{tabular}
	\label{tab:exemple-calcul-1}
\end{table} Ces résultats nous permettent de voir que, été comme hiver, une même
variation d'exergie spécifique apportée à l'air humide le sera avec une
variation de température plus faible, relativement, que la variation de pression
correspondante. Dis autrement, eu égard aux pertes de charges typiquement
rencontrées dans les systèmes de traitement d'air, ces dernières produisent des
variations d'exergie spécifique très inférieures à celles générées par les
variations de température usuelles auxquelles l'air est soumis. C'est pour cette
raison que la plupart des études traitant d'applications pratiques de l'analyse
exergétique aux systèmes de traitement de l'air font explicitement l'hypothèse
que l'exergie spécifique mécanique \eqref{eq:exergie-m-1} est négligeable devant
les autres composants de l'exergie spécifique \eqref{eq:exergie-specifique}, et ne traitent que de ses fractions thermiques et chimiques.

%Dans le cas d'un bâtiment par exemple, la nécessité de chauffer ou de refroidir
%l'air de ventilation, de l'humidifier ou de le déshumidifier, dépend de la
%température et de l'humidité extérieure. Réciproquement à la définition
%précédente, l'exergie est aussi la quantité minimale de travail qu'il faut
%échanger avec une quantité ou un flux d'énergie, pour les faire passer d'un état
%d'équilibre avec le milieu extérieur à un état d’état d'énergie plus élevée. Par
%état, on entend en pratique la température, la pression et la composition
%chimique.

%\textsl{Les applications de traitement de l'air constituent un exemple très
%intéressant d'utilisation de l'analyse exergétique parce qu'elles consomment
%justement de l'air venant de l’extérieur, et dont l'exergie spécifique est
%nulle par définition.}

\subsection{Performances exergétiques des systèmes de traitement de l'air}

\section{Les pièges de l'analyse exergétique}

\subsection{Connaissance de l'état de référence}

\subsection{Influence des incertitudes de mesures}

\subsection{Le problème de l'état de référence de l'eau}

\section{Conclusion}

\section*{Notations} \label{table-notations}

\begin{multicols}{2}
	\begin{tabbing}
		\hspace{0.05\textwidth} \= \hspace{0.45\textwidth} \kill
		$c_p$ \> Capacité thermique spécifique à \\
		\> pression constante, en $\mathrm{J/(kg \cdot K)}$. \\
		$C$ \> Capacité thermique, en $\mathrm{J/K}$. \\
		$ex$ \> Exergie spécifique massique, en $\mathrm{J/kg}$. \\
		$Ex$ \> Exergie, en $\mathrm{J}$. \\
		$m$ \> Masse, en $\mathrm{kg}$. \\
		$M$ \> Mass molaire, en $\mathrm{kg/m}^3$. \\
		$n$ \> Quantité molaire, en $\mathrm{mol}$. \\
		$p$ \> Pression, en $\mathrm{bar}$. \\
		$\mathsf{r}$ \> Constante spécifique massique d'un \\
		\> gaz parfait. \\
		$\mathsf{R}$ \> Constante universelle des gaz parfaits. \\
		$T$ \> Température, en $\degree\mathrm{C}$. \\
		\> \textbf{Alphabet grec} \\
		$\mu$ \> Potentiel chimique, en $\mathrm{J/mol}$. \\
		$\omega$ \> Humidité spécifique, définie par
		\eqref{eq:def-omega}. \\
		$\Theta$ \> Facteur de \textsc{Carnot}. \\
		\> \textbf{Indices} \\
		$0$ \> Milieu extérieur au système. \\
		$a$ \> Air sec. \\
		ml \> Moyenne logarithmique. \\
		$v$ \> Vapeur d'eau. \\
	\end{tabbing}
\end{multicols}

\printbibliography

\end{document}
